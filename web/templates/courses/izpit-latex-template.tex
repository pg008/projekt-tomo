
\documentclass{izpit}  % naložite iz https://github.com/ul-fmf/izpit
% Paket lahko naložite z neobveznimi možnostmi:
% - arhiv:
%     Izpis, namenjen za objavo izpita v arhivu. Naloge se pišejo ena pod
%     drugo, brez vmesnega prostora za reševanje, v glavi pa ni vpisnih polj.
% - izpolnjen:
%     Izpis, namenjen za generirane domače naloge. V glavi ni vpisnih polj,
%     saj se predpostavlja, da so že izpolnjena (vsak študent ima svojo
%     verzijo)
% - brezpaketov:
%     Prepreči nalaganje paketov amsmath, amssym, babel in inputenc. To
%     možnost uporabite, če je kakšen od teh paketov v konfliktu z vašimi.
%     Paketi ifthen, keyval, geometry in tikz se vedno naložijo, saj so
%     za uporabo paketa izpit obvezni.
% - sumniki:
%     Če vaš urejevalnik pod Windowsi ne pozna kodne tabele UTF-8
%     (ostali operacijski sistemi že leta podpirajo UTF-8),
%     uporabite možnost sumniki, da boste lahko pisali č namesto "c, ali \v c.
% - 10pt, 11pt, fleqn, ...
%     Uporabite lahko tudi vse ostale možnosti ki obstajajo v paketu article.
%     V osnovi je velikost črk 11pt.

\usepackage{hyperref}
\usepackage{minted}
\setminted{fontsize=\small, autogobble, breaklines, breakanywhere, frame=lines, framesep=2mm}
\newmintinline[py]{Python}{}
\BeforeBeginEnvironment{minted}{\vspace{-14pt}}
% \AfterEndEnvironment{minted}{\vspace{-14pt}}
\newcommand{\Primer}{\vspace{\baselineskip} \noindent Primer:}

\begin{document}

% Vsak izpit se začne z ukazom \izpit{predmet}{datum}{pravila}.
% Če želite v eni datoteki ustvariti več izpitov, lahko ukaz uporabite večkrat.
%
% Ukaz \izpit sprejme naslednje neobvezne možnosti:
% - ucilnica:
%     V glavo izpita se natisne shema učilnice, na kateri študentje lahko
%     označijo svoj sedež. Na voljo so učilnice: 201, 202, 203, 204, 205,
%     304, 305, 306, 307, 310, 311, 312, P01, P02, P04, P05, F1, F2, MFP in VFP.
%     Povejte, katere učilnice bi še potrebovali. Za idejo in pomoč pri izdelavi
%     se zahvaljujem Alešu Vavpetiču, Jaki Smrekarju in Janošu Vidaliju.
% - sedezni red:
%     Na shemi učilnice so dovoljeni sedeži odebeljeni.
% - naloge:
%     V glavo izpita se natisnejo okenca za vpis točk posameznih nalog ter
%     skupnega števila točk. Če je število nalog enako 0, se natisne samo
%     okence skupnega števila točk, če pa je negativno, se ne natisne nič. V
%     osnovi se natisnejo okenca za 4 naloge ter skupne točke.
% - anglescina:
%     Če želite izpit sestaviti v angleščini, uporabite parameter
%     'anglescina'. Prevedena bodo polja za ime, vpisno številko, oznake nalog
%     ter števila točk (če ste uporabili ukaz \tocke{} - glej spodaj).
% - nadaljuj:
%     Če izpit obsega več listov, lahko ukaz \izpit ponovimo na novem listu,
%     da se ponovno izpiše glava. Parameter 'nadaljuj' poskrbi za to, da se
%     števec nalog ne ponastavi in se tako številčenje nadaljuje na novem listu.
% - brez vpisne:
%     Polje za vnos vpisne številke se ne izriše.

\izpit[ucilnica = MFP, naloge = {{ problem_set.visible_problems|length }}]
  {FIXME {{ problem_set.course.title }}: {{ problem_set.title }}}{FIXME }{
  Čas pisanja je TODO minut. Možno je doseči TODO točk, od tega TODO za slog kode. Veliko uspeha!
}

% Nalogo začnete z ukazom \naloga, podnalogo pa z ukazom \podnaloga. Oba ukaza
% sprejmeta neobvezen parameter, v katerega zapišete število točk.
% Na voljo vam je ukaz \tocke, ki sprejme število ter na konec doda besedo
% 'točke' (oz 'marks') s primerno končnico: \tocke{5} zapiše 5 točk,
% \tocke{102} zapiše 102 točki, ...

\naloga[{{ problem.title|md2tex }}, \tocke{TODO}]
{{ problem.description|md2tex }}


\podnaloga[TODO] {{ part.description|md2tex }}




\end{document}
